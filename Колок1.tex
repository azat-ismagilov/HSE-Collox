\documentclass[12pt,a4paper]{article}
\usepackage[utf8]{inputenc}
\usepackage[russian]{babel}
\usepackage[OT1]{fontenc}
\usepackage{amsmath}
\usepackage[margin=0.7in]{geometry}
\usepackage{amsfonts}
\renewcommand{\baselinestretch}{1.2}
\usepackage{amssymb}
\title{Коллоквиум по мат. анализу №1}
\begin{document}
\maketitle
\section{Билет}
\begin{itemize}
\item \textbf{Рациональные числа} - числа вида $\frac{p}{q}$, где $q$ - натуральное число, а $p$ - целое. Считается, что две записи $\frac{p_1}{q_1}$ и  $\frac{p_2}{q_2}$ задают одно и то же рациональное число, если $p_1q_2=p_2q_1$. Обратим внимание на то, что рациональных чисел не достаточно для естественных потребностей математики.

\item \textbf{Вещественные числа} - множество всех бесконечно десятичных дробей вида $\pm a_0a_1a_2...$, где $a_0 \in N \vee {0}, a_j \in {0...9}$ (Записи, в которых с какого-то момента стоят только 9-ки запрещены); \\
Число $\pm 0,000...$ называется нулём и совпадает с числом 0;\\
Нунелевое число: \\
- положительное, если в его записи стоит знак '+'; \\
- отрицательное, если в его записи стоят знак '-'; \\
В вещественные числа вложены рациональные естественным образом. У вещественных чисел также определены операции сложения и умножения для которых справедливы все их естественные свойства. \\
Отношение порядка у вещественных чисел задано лексикографическим порядком ($a_0a_1a_2...\leq b_0b_1b_2... \; \exists k: a_0 = b_0, ..., a_{k-1}=b_{k-1}, a_k \leq b_k$), который естественным обращом переносится на отрицательные. \\
Для вещественных чисел определён модуль числа $a$, т.е. такое вещественное число, что $|a| = a$, если $a \geq 0$ и $|a| = -a$, если $a < 0$. Также, для модуля выполняется неравенство треугольника $|a+b| \leq |a| + |b|$. Из неравенства треугольника следует, что $||a|-|b|| \leq |a+b|$. \\
Самое важное свойство - выполняется принцип полноты;

\item \textbf{Десятичные дроби.} Рациональное число может быть представлено в виде конечной или периодической десятичной дроби ($\frac{1}{10} = 0.1; \frac{1}{6} = 0.1(6); \frac{1}{7} = 0.(142857)$. Можно не рассматривать десятичные записи с периодом 9, т.к. $0.(9) = 1$ (Если $0.(9) = x$, то $10x=9+x$ - истина, октуда $x = 1$.

\item \textbf{Принцип полноты. }
Принцип полноты выполняется, если для произвольных непустых множеств $A$ левее $B$ найдется разделяющий их элемент. \\
Принцип полноты не выполняется для рациональных чисел. \\
Принцип полноты выполняется на множестве вещественных чисел (теорема).\\
\textbf{Доказательство:} \\
Пусть $A$ и $B$ - непустые множества.  $A$ левее $B$. Если $A$ состоит только из неположительных чисел, а $B$ только из неотрицательных, то нуль разделяет $A$ и $B$. Пусть в $A$ имеется положительный элемент, тогда $B$ состоит только из положительных чисел (обратный случай аналогичен). Построим число $c = c_0c_1c_2...$, разделяющее $A$ и $B$. \\
Рассмотрим множество натуральных чисел, с которых начинаются элементы множества $B$. Пусть $b_0$ - наименьшее из таких и пусть $b_0 = c_0$. Затем рассмотрим все числа в множестве $B$, начинающиеся с $b_0$ и найдем у них наименьшую первую цифру после запятой и предположим, что $b_1=c_1$ (где $b_1$ - эта цифра). Теперь рассмотрим все числа в в множестве $B$, начинающиеся с $b_0.b_1$ и найдем у них наименьшую вторую цифру после запятой $b_2$, тогда пусть $c_2=b_2$  и т.д. получим бесконечную десятичную дробь $c_0c_1c_2...$. Покажем, что построенное число рязделяет множества $A$ и $B$. Во-первых, по построению $c \leq b$ для каждого $b \in B$. Действительно, либо $b = c$, либо $b \neq c$. Во втором случае пусть $b_0 = c_0, ..., b_{k-1} = c_{k-1}$ и $b_k \neq c_k$. Тогда по построению числа $c$, $c_k < b_k \Rightarrow c < b$. \\
Покажем, что для каждого $a \in A $ $a \leq c$. Предположим, что $a > c$, т.е. $a \geq c$ и $a \neq c$. Тогда найдется позиция $k$, для которой $a_0=c_0, ..., a_{k-1}=c_{k-1}$ и $a_k > c_k$. Но по построению числа $c$ есть такой $b \in B$, что $b_0 = c_0, ..., b_k=c_k$, значит $a_k > b_k$, что противоречит условию $A$ левее $B$. 

\item \textbf{Рациональных решений уравнение $x^2=2$ не существует. }\\
Действительно, пусть $\frac{p}{q}$ - такое решение и $p$ и $q$ не имеют общих делителей. Тогда $\frac{p^2}{q^2} = 2 \Rightarrow p^2=2q^2 \Rightarrow p^2$ - четное $\Rightarrow$ p - четное $\Rightarrow p=2k, 4k^2=2q^2 \Rightarrow 2k^2 = q^2 \Rightarrow q^2$ - четное $\Rightarrow q$ - четное $\Rightarrow$ числа $p$ и $q$ имеют общий делитель. Противоречие.

\item \textbf{Существования $\sqrt2$ как следствие принципа полноты} \\
Пусть $A=\{a: a > 0, a^2 \leq 2\}$ и $B = \{b: b > 0, b^2 \geq 2\}$. Заметим, что множество $A$ лежит левее множества $B$, т.к. $0 < b^2 - a^2 = (b-a)(b+a)$ для каждых $a \in A, b \in B$, и $a + b > 0$. Если бы существовало число $c$, разделяющее множества $A$ и $B$, то обязательно $c^2 = 2$. Действительно, во-первых, сразу заметим, что $1 \leq c \leq 2$, т.к. $1 \in A, 2 \in B$. Теперь, если $c^2 < 2$, то число $c + \frac{2 - c^2}{5} \in A$, т.к.
\[
(c + \frac{2 - c^2}{5})^2 = c^2 + 2c*\frac{2 - c^2}{5} + (\frac{2 - c^2}{5})^2 \leq c^2 + 4*\frac{2 - c^2}{5} + \frac{2 - c^2}{5} \leq 2.
\]
Но $c+\frac{2-c^2}{5} > c$, а значит $c$ не разделяет $A$ и $B$.\\
Если же $c^2 > 2$, то $c - \frac{c^2-2}{4} \in B$, т.к.
\[
(c + \frac{c^2 - 2}{4})^2 \geq c^2 - 2c*\frac{c^2-2}{4} \geq c^2 - 4 * \frac{c^2 - 2}{4} = 2
\]
Но $c - \frac{c^2 - 2}{4} < c$, а значит $c$ не разделяет $A$ и $B$. \\
Тем самым, $c^2 = 2$.
 
\end{itemize}

\section{Билет}
\begin{itemize}
    \item \textbf{Предел последовательности} \\
    Если каждому числу $n \in N$ поставлено в соответствие некоторое число $a_n$, то говорим, что задана числовая последовательность $\{a_n\}_{n=1}^{\infty}$ \\
    Говорят, что последовательность $\{a_n\}_{n=1}^{\infty}$ сходится к числу $a$, если для каждого $\varepsilon > 0$ найдется такой номер $N_{\varepsilon} \in N$, что $|a_n - a| < \varepsilon$ при каждом $n > N_{\varepsilon}$. \\
    $\forall \varepsilon > 0 \; \exists N_{\varepsilon} \in N : \forall n> N_{\varepsilon} |a_n - a| < \varepsilon$ \\
    $ \lim_{n\to\infty} a_n \ = a$  или $a_n \to a$ при $n \to \infty$
    \item \textbf{Единственность предела}
    Пусть  $\lim_{n \to \infty} a_n\ = a$ и $\lim_{n \to \infty} a_n\ = b$, тогда a = b. \\
    \textbf{Доказательство:} Если $a \neq b$, то $|a - b| = \varepsilon_0 > 0$. Но по определению найдется номер $N_1$, для которого $|a_n - a| < \frac{\varepsilon_0}{2}$ при $n > N_1$ и найдется номер $N_2$, для которого $|a_n - b| < \frac{\varepsilon_0}{2}$ при $n > N_2$. Тогда при $n > max \{N_1, N_2\} : \varepsilon_0 = |a - b| = |a - a_n + a_n - b| \leq |a - a_n| + |a_n - b| < \varepsilon_0$. Противоречие.
    \item \textbf{Арифметика предела.}
    $\lim_{n \to \infty} a_n\ = a$ и $\lim_{n \to \infty} b_n\ = b$\\
    1) $ \lim_{n \to \infty} (\lambda a_n + \beta b_n) = \lambda a + \beta b \;  \forall a, b \in R $ \\
    2) $\lim_{n \to \infty} a_n b_n\ = ab$\\
    3) Если $b \neq 0, b_n \neq 0$, то $\lim_{n \to \infty} \frac{a_n}{b_n}\ = \frac{a}{b}$. \\
    \textbf{Доказательство:} Пусть $\varepsilon > 0$ - произвольное число. Тогда найдется номер $N_1$, для которого $|a_n - a| < \varepsilon$, и найдется номер $N_2$, для которого $|b_n - b| < \varepsilon$\\
    1) При $n > N = max\{ N_1, N_2 \} : |\lambda a_n + \beta b_n - (\lambda a + \beta b)| = |\lambda(a_n - a) + \beta(b_n - b)| \leq |\lambda| |a_n - a| + |\beta| |b_n - b| < (|\lambda| + |\beta|)\varepsilon$ \\
    2) Заметим, что $|a_n b_n - a b| = |a_n b_n - a b_n + a b_n - a b| \leq |b_n| |a_n - a| + |a| |b_n - b|$. Т.к. сходящаяся последовательность ограничена, то найдется $M > 0$, для которого $|b_n| \leq M$, поэтому при  $n > N = max\{ N_1, N_2\}$ выполнено $|a_n b_n - a b| \leq (M + |a|)\varepsilon$\\
    3) Достаточно проверить, что $\frac{1}{b_n} \to \frac{1}{b}$ при $n \to \infty$. Заметим, что по условию $b \neq 0$, поэтому найдется номер $N_3 \in N$, для которого при $n > N_3$ выполнено $|b_n| > \frac{|b|}{2}$. Тогда при $n > max \{N_2, N_3\}$ выполнено \[|\frac{1}{b_n} - \frac{1}{b}| = \frac{|b_n - b|}{|b_n| |b|} \leq \frac{2}{|b|^2} * \varepsilon\]
    \item \textbf{Ограниченность сходящейся последовательности}:\\
    Утверждение: сходящаяся последовательность ограничена\\
    \textbf{Доказательство:} Если $\lim_{n \to \infty} a_n\ = a$, то для некоторого $N \in \mathbb{N}$ выполнено  
    $|a_n - a| < 1$ при $n > N  \Rightarrow |a_n| = |a_n - a + a| \leq |a_n - a| + |a| < 1 + |a|$ при $n > N$.
    Значит $|a_n| \leq M = max\{1 + |a|, |a_1|, |a_2|, ..., |a_N|\}$, т. е. $-M = c \leq a_n \leq C = M$.
    
	\item \textbf{Свойство $|a_n - a| \leq \alpha\varepsilon$} \\
	Пусть $\alpha > 0$ - фиксированное число. Тогда свойство
	\[
	\forall \varepsilon > 0 \; \exists N(\varepsilon) \in \mathbb{N}: \forall n > N(\varepsilon) \; |a_n - a| \leq \alpha \varepsilon	
	\]
	равносильно тому, что $\lim_{n \to \infty}{a_n} = a$. Действительно, пусть $\varepsilon'$ - проиозвольное число, тогда найдется номер $N(\frac{\varepsilon'}{2\alpha}) \in \mathbb{N}$, для которого $|a_n - a| \leq \alpha * \frac{\varepsilon'}{2\alpha} < \varepsilon'$ при каждом $n > N(\frac{\varepsilon'}{2\alpha})$. В другую сторону доказательство аналогично.
    
    \item \textbf{Отделимость}:
    Если $a_n \to a$ и $a \neq 0$, то найдется номер $N \in \mathbb{N}$, для которого $|a_n| > \frac{|a|}{2} > 0$ при $n > N$.\\
    \textbf{Доказательство:} Взяв $\varepsilon = \frac{|a|}{2}$ в определении сходимости последовательности к числу $a$, получаем номер $N \in \mathbb{N}$, для которого $|a_n - a| < \frac{|a|}{2}$ при $n > N$. Тогда при $n > N$, выполнено $|a| - |a_n| \leq |a_n - a| < \frac{|a|}{2}$, что равносильно тому, что мы доказываем.
  
\end{itemize}
\section{Билет}
\begin{itemize}
\item \textbf{Переход к пределу в неравенствах} \\
Пусть $\lim_{n \to \infty}{a_n = a}, \lim_{n \to \infty}{b_n = b}$. Если для некоторого номера $N$ выполнено $a_n \leq b_n$ при $n > N$, то и $a \leq b$; \\
\textbf{Доказательство:} Предположим, что $a-b = \varepsilon > 0$. Тогда найдутся номера $N_1 \in N$ и $N_2 \in N$, для которых $|a_n - a| < \frac{\varepsilon}{2}$ при $n > N_1$, и $|b_n-b| < \frac{\varepsilon}{2}$ при $n > N_2$. Тогда $\varepsilon = a - b = a - a_n + a_n - b_n + b_n - b \leq a - a_n + b_n - b < \varepsilon$. Противоречие. 

\item \textbf{ Лемма о зажатой последовательности} \\
Пусть $\lim_{n \to \infty}{a_n} = \lim_{n \to \infty}{b_n} = b$ и для некоторого $N \in \mathbb{N}$ выполнено неравенство: $a_n \leq c_n \leq b_n$ при $n > N$. Тогда $\lim_{n \to \infty}{c_n = a}$. \\
\textbf{Доказательство:} Для каждого $\varepsilon > 0$ найдутся номера $N_1 \in \mathbb{N}$ и $N_2 \in \mathbb{N}$, для которых $|a_n - a| < \varepsilon$ и $|b_n - a| < \varepsilon$. Тогда при $n > max\{N, N_1, N_2\}$ выполнено: $a - \varepsilon < a_n \leq c_n \leq b_n < a + \varepsilon$.

\item \textbf{Принцип вложенных отрезков} \\
Пусть $a, b \in R$ и $a < b$. Множества $[a; b] := \{x \in R: a \leq x \leq b\}$, $(a;b) := \{ x \in R: a < x < b \}$ называются отрезком и интервалом соответственно. Длиной отрезка (интервала) называется величина $b - a$. \\
\textbf{Теорема:} Всякая последовательность $\{[a_n, b_n]\}_{n=1}^{\infty}$ вложенных отрезков (т.е. $[a_{n+1};b_{n+1}] \subset [a_{n};b_{n}]$) имеет общую точку. Кроме того, если длины отрезков стремятся к нулю, т.е. $b_n - a_n \longrightarrow 0$, то такая общая точка только одна. \\
\textbf{Доказательство:} по условию $[a_{n+1}; b_{n+1}] \subset [a_{n}; b_{n}]$, откуда $a_{n} \leq a_{n+1} \leq b_{n+1} \leq b_n$. Заметим, что при $n < m$ выполнено $a_n \leq a_m \leq b_m$, а при $n > m$ выполнено $a_n \leq b_n \leq b_m$. Таким образом, если $A :=\{a_n, n \in \mathbb{N}\}$ и $B := \{b_m, m \in \mathbb{N}\}$, то $A$ левее $B$, а значит по принципу полноты найдется такое число $c \in \mathbb{R}$, что $a_n \leq c \leq b_m$ для произвольных $n, m \in N$. В частности $a_n \leq c \leq b_n$ т.е. $c \in [a_n; b_n]$. \\
Пусть общих точек две: $c$ и $c'$. Не ограничивая общности $c < c'$. Тогда $a_n \leq c < c' \leq b_n$ и $c' - c \leq b_n - a_n$, что противоречит тому, что $\lim_{n \to \infty}{(b_n - a_n)} = 0$. Действительно, найдется номер $N$, для которого $b_n - a_n < c' - c$ при каждом $n > N$.

\item \textbf{Геометрическая интерпретация вещественных чисел, вещественная прямая.} \\
Доказанная выше теорема позволяет дать вещественным числам следующую геометрическую интерпретацию. Сопоставим десятичной дроби $0.a_1a_2...$ последовательность вложенных отрезков по следующему правилу: \\
Разделим отрезок $[0;1]$ на 10 равных частей и выберем из получившихся 10 отрезков $(a_1 + 1)$-ый по счету. Делаем то же самое и выбираем $(a_2 + 1)$-ый по счету и т.д. Получаем последовательность вложенных отрезков, причем длина отрезка на $n$-ом шаге равна $10^{-n}$. По доказанной теореме сущестует единственная общая точка построенной последовательности вложенных оторезков, которая как раз и совпадает с $0.a_1a_2...$
\end{itemize}

\section{Билет}
\begin{itemize}
	\item \textbf{Точные верхние и нижние грани.} \\
	Пусть $A$ - непустое подмножество вещественных чисел.
	Число $b$ называется верхней гранью множества $A$, если $a \leq b$ верно для каждого числа $a \in A$. Если есть хоть бы одна верхняя грань, то множество называют ограниченным сверху. Наименьшая из верхних граней множества $A$ называется точной верхней гранью множества $A$ и обозначается $sup(A)$ (супремум).\\
	Число $b$ называется нижней гранью множества $A$, если $b \leq a$ верно для каждого числа $a \in A$. Если есть хотя бы одна нижняя грань, то множество называется ограниченным снизу. Наибольшая из нижних граней множества $A$ называется точной нижней гранью множества  $A$ и обозначается $inf(A)$ (инфинум)\\
	Ограниченное и сверху, и снизу множество называется ограниченным.
	
\item \textbf{Теорема Вейерштрассе о пределе монотонной ограниченной последовательности.} \\
	Пусть последовательность $\{a_n\}_{n=1}^{\infty}$ не убывает $(a_n \leq a_{n + 1})$ и ограничена сверху. Тогда эта последовательность сходится к своему супремуму.\\
	Анологично, пусть последовательность $\{a_n\}_{n=1}^{\infty}$ не возрастает $(a_{n + 1} \leq a_n)$ и ограничена снизу. Тогда эта последовательность сходится к своему инфинуму.\\
	\textbf{Доказательство:} Пусть $M = \sup\{a_n : n \in \mathbb{N}\} = \sup(a_n)$. Тогда для каждого $\varepsilon > 0$ найдется номер $N \in \mathbb{N}$, для которого $M - \varepsilon < a_N$, (иначе $M - \varepsilon$ - верхняя грань, чего не может быть. В силу того, что последовательность неубывающая, при каждой $n > N$ выполнено $M - \varepsilon < a_N \leq a_n \leq M < M + \varepsilon $\\
	Тем самым, по определению $M = \lim\ a_n$\\
	Случай с невозрастающей последовательностью рассматривается аналогично.
	
\item \textbf{Пример рекуррентной формулы для вычисления $\sqrt2$\\}
	Пусть $a_{n + 1} = \frac{1}{2} (a_n + \frac{2}{a_n}), a_1 = 2$. Заметим, что
	 \[a_{n + 1} = \frac{1}{2} (a_n + \frac{2}{a_n}) \geq \frac{1}{2} * 2 \sqrt{(a_n * \frac{2}{a_n})} = \sqrt2\] 
	Поэтому $a_n \geq \sqrt2$. Кроме того, $a_{n + 1} = \frac{1}{2} (a_n + \frac{2}{a_n}) \leq \frac{1}{2} (a_n + \frac{a_n^2}{a_n}) = a_n$. По доказанной Теореме у последовательности $\{a_n\}_{n=1}^{\infty}$ существует предел $a$. Т. к. $a_n \geq 0$, то и $a \geq 0$.\\
	Тогда по арифметике предела получаем $a = \frac{1}{2} (a + \frac{2}{a})$, откуда $a = \sqrt2$.
	
	\textbf{Скорость сходимости}
	\[
	|a_{n+1} - \sqrt2| = \frac{|a_{n}^2 - 2a_n\sqrt2 + 2|}{2a_n} = \frac{(a_n - \sqrt2)^2}{2a_n} \leq  \frac{(a_n - \sqrt2)^2}{2\sqrt2} \leq (a_n - \sqrt2)^2;
	\]
	Индуктивно получаем:
	\[
	|a_{n+1} - \sqrt2| \leq (a_n - \sqrt2)^2 \leq (a_{n-1} - \sqrt2)^4 \leq (a_{n-2} - \sqrt2)^8 \leq (a_1 - \sqrt2)^{2^{n+1}} = (2-\sqrt2)^{2^{n+1}};	
	\]
	Заметим, что $q := 2 - \sqrt2 < 1$, поэтому полученная скорость сходимость $q^{2^n}$ быстрее экспоненциальной $q^n$ (в смысле количества применений рекуррентной формулы для достижения заданной точности)

\end{itemize}

\section{Билет}
\begin{itemize}

\item \textbf{Фундаментальная последовательность} \\
Говорят, что последовательность $\{a_n\}_{n=1}^{\infty}$ фундаментальная (или является последовательностью Коши), если для каждого числа $\varepsilon > 0$ найдется такое натуральное число (номер) $N(\varepsilon) \in \mathbb{N}$, что $|a_n - a_m| < \varepsilon$ при каждом $n, m > N(\varepsilon)$. То же самое в кванторах: $\forall \varepsilon > 0 \; \exists N(\varepsilon) \in \mathbb{N}: \forall n, m > N(\varepsilon) \; |a_n - a_m| < \varepsilon$.
[Если последовательность $\{a_n\}_{n=1}^{\infty}$ сходится, то она фундаментальная]

\item \textbf{Критерий Коши: }Числовая последовательность сходится тогда и только тогда, когда она фундаментальна \\
Это следует из того, что: \\
1) Если последовательность сходится, то она фундаментальная; \\
2) Если последовательность фундаментальная, то она сходится; \\
\textbf{Доказательство:} \\
1) Пусть $\varepsilon > 0$. По определению сходящейся последовательности найдется такой номер $N \in \mathbb{N}$, что $|a_n - a| < \frac{\varepsilon}{2}$ при $n > N$, где $a := \lim_{n \to \infty}{a_n}$. Тогда при $m, n > N$ выполнено:
\[
|a_n - a_m| = |a_n - a + a - a_m| \leq |a_n - a| + |a_m - a| < \varepsilon
\]
2) Заметим, что последовательность $\lim_{n \to \infty}{a_n}$ ограничена. Действительно, для некоторого $N \in \mathbb{N}$ выполнено $|a_n - a_m| < 1$ при $m, n > N$. Отсюда при $n > N$:
\[
|a_n| = |a_n - a_{N+1} + a_{N+1}| \leq |a_n - a_{N+1}| + |a_{N+1}| < 1 + |a_{N+1}|
\]
Значит, $|a_n| \leq M = max\{1 + |a_{N+1}|,|a_1|, ..., |a_N|\}$ \\
Пусть $M_n := \sup_{k > n}{a_k}$. Тогда последовательность $\{M_n\}_{n=1}^{\infty}$ не возрастает, т.к. $M_n$ является верхней гранью для множества $\{a_k: k > n+1\}$, т.е. с ростом n количество значений, из которых берется супремум, не увеличивается. Кроме того, т.к. все $a_k \geq -M$, то и $M_n \geq -M$. Таким образом, $\exists a = \lim_{n \to \infty}{M_n}$. 

Покажем, что $a_n \longrightarrow a$. Пусть $\varepsilon > 0$, тогда найдется номер $N$, для которого $|a_n - a_m| < \varepsilon$ при $m, n > N$. Кроме того, найдется номер $N_1$, для которого $|M_n - a| < \varepsilon$ при $n > N_1$. Пусть $N_2 = max\{N, N_1\}$. Найдется номер $m > N_2$, для которого $M_{N_2+1} - \varepsilon < a_m \leq M_{N_2+1}$. Тогда, при $n > N$: \\
$|a_n - a| = |a_n - a_m + a_m - M_{N_2+1} + M_{N_2+1} - a| \leq |a_n - a_m| + |a_m - M_{N_2+1}| + |M_{N_2+1} - a| < \varepsilon + \varepsilon + \varepsilon$

\item \textbf{Пример применения критерия Коши для доказательства представления $\sqrt2$ цепной дробью.} \\
Пусть $a_{n+1} = 1 + \frac{1}{1+a_n}, a_1 = 1$. Заметим, что $a_n \geq 1$ и:
\[
|a_{n+1} - a_n| = |\frac{1}{1 + a_n} - \frac{1}{1 + a_{n-1}}| = \frac{|a_n - a_{n-1}|}{(1+a_n)(1+a_{n-1})} \leq \frac{1}{4} |a_n - a_{n-1}| \leq \\
\]
\[
\leq (\frac{1}{4})^{n-1} |a_2 - a_1| = (\frac{1}{4})^{n-1}*\frac{1}{2} \Rightarrow \\
\]
$\Rightarrow$ Отсюда при $m > n$ \\
\[
|a_m - a_n| \leq |a_m - a_{m-1}| + ... + |a_{n+1} - a_n| \leq \frac{1}{2} ((\frac{1}{4})^{m-2} + ... + (\frac{1}{4})^{n-1}) = \\
\]
\[
= \frac{1}{2}(\frac{1}{4})^{n-1} \frac{1 - (\frac{1}{4})^{m - n}}{1 - \frac{1}{4}} \leq \frac{8}{3}(\frac{1}{4})^n
\]
Т.к. $(\frac{1}{4})^n \longrightarrow 0$, то для каждого $\varepsilon > 0$ найдется $N$, для которого $(\frac{1}{4})^n < \varepsilon$ при $n > N$. Тем самым, для последовательности $\{a_n\}_{n=1}^{\infty}$ выполнен критерий Коши, а, значит, существует $a = \lim_{n \to \infty}{a_n}$. По арифметике предела $a$ удовлетворяет уравнению: \[a(1+a) = 1 + a +1 \Leftrightarrow a^2 = 2 \Leftrightarrow a = \sqrt2,\]
т.к. $a \geq 0$.
\end{itemize}

\section{Билет}
\begin{itemize}
  \item \textbf{Числовые ряды.}\\
  Пусть $\{a_n\}_{n=1}^{\infty}$ - числовая последовательность. Числовым рядом с членами $a_n$ называется выражение $a_1 + a_2 + a_3 + ... =  \sum_{k = 1}^{\infty} a_k$.\\
  Конечные суммы $S_n := \sum_{k = 1}^n a_k$ называют частичными суммами ряда $\sum_{k = 1}^{\infty} a_k$. Говорят, что ряд $\sum_{k = 1}^{\infty} a_k$ сходится, если у последовательности $\{S_n\}_{n = 1}^{\infty}$ существует предел, который называют суммой ряда. Если такого предела не существует, то говорят, что ряд расходится (не сходится).\\
  В силу арифметики предела на сходимость ряда (но не на сумму) не влияет добавление или отбрасывание первых нескольких слагаемых.

  \item \textbf{Переформулировка критерия Коши для числовых рядов}:\\
  Ряд $\sum_{k = 1}^{\infty} a_k$ сходится тогда и только тогда, когда для каждого $\varepsilon > 0$ найдется такой номер $N$, что для всех $n > m > N$ выполнено $|\sum_{k = m + 1}^n a_k| = |S_n - S_m| < \varepsilon$.
  
  \item \textbf{Необходимое условие сходимости числового ряда:}\\
  Если ряд $\sum_{k = 1}^{\infty} a_k$  сходится, то $a_k \to 0$ при $k \to \infty$
  
  \textbf{Доказательство:} Действительно, из критерия Коши следует, что для каждого $\varepsilon > 0$ найдется номер $N$, для которого при каждом $n > N + 1$ выполнено $|a_n| = |S_n - S_{n - 1}| < \varepsilon$.
  
  \item \textbf{Расходимость ряда $\sum_{n = 1}^{\infty}{\frac{1}{k}}$:}\\
  Рассмотрим такой ряд. Исследуем его на выполнение условия критерия Коши: пусть $n > m$, тогда 
\[
  |\sum_{k = m + 1}^n \frac{1}{n}| \geq \frac{n - m}{n}
\]\\
  Какой бы ни был задан номер $N$ всегда можно взять $m > N$ (например, $m = N + 1$ ) и $n = 2m$, тогда 
\[
  |\sum_{k = m + 1}^{n = 2m} a_k| \geq \frac{1}{2},
\]\\
а значит условие критерия Коши не выполнено и ряд расходится.
  
\item \textbf{Абсолютная и условная сходимость рядов.} \\
Говорят, что ряд $\sum_{k = 1}^{\infty} a_k$ сходится абсолютно, если сходится ряд $\sum_{k = 1}^{\infty} |a_k|$.\\
Говорят, что ряд $\sum_{k = 1}^{\infty} a_k$ сходится условно, если он сходится, а ряд $\sum_{k = 1}^{\infty} |a_k|$ расходится.
\end{itemize}

\section{Билет}
\begin{itemize}
  \item \textbf{Сходимость рядов с неотрицательными слагаемыми + признак сравнения.}\\ 
  Пусть $a_k \geq 0$, тогда ряд $\sum_{k = 1}^{\infty} a_k$ сходится тогда и только тогда, когда последовательность его частичных сумм ограничена.
  
  \textbf{Доказательство:} утверждение следует из того, что последовательность частичных сумм не убывает. \\
  Отсюда получаем такой признак сравнения.\\
  Пусть $0 \leq a_n \leq b_n$. Если ряд $\sum_{k = 1}^{\infty} b_k$ сходится, то сходится и ряд $\sum_{k = 1}^{\infty} a_k$.\\
  Наоборот, если ряд $\sum_{k = 1}^{\infty} a_k$ расходится, то расходится и ряд $\sum_{k = 1}^{\infty} b_k$.\\
  \item \textbf{Признак Коши}\\
  Пусть $\{a_n\}_{n = 1}^{\infty}$ - невозрастающая последовательность, $a_n \geq 0$. Ряд $\sum_{k = 1}^{\infty} a_k$ сходится тогда и только тогда, когда сходится ряд $\sum_{k = 1}^{\infty} 2^k a_{2^k}$.
  
  \textbf{Доказательство:} Заметим, что $a_2 + 2a_4 + 4a_8 + ... + 2^{n - 1} a_{2^n} \leq a_2 + a_3 + a_4 + a_5 + ... + a_{2^n} \leq 2a_2 + 4a_4 + 8a_8 + ... + 2^n a_{2^n}$. \\
Отсюда получаем, что из ограниченности частичных сумм ряда $\sum_{k = 1}^{\infty} 2^k a_{2^k}$ следует ограниченность частичных сумм ряда $\sum_{k = 1}^{\infty} a_k$ и наоборот.
\\
\item \textbf{Сходимость и расходимость рядов $\sum_{k = 1}^{\infty} \frac{1}{k^p}$ в зависимости от $p$.}\\
Ряды $\sum_{k = 1}^{\infty} \frac{1}{k^p}$ сходятся при $p > 1$ и расходятся при $p \leq 1$.\\
При $p \leq 0$ слагаемое $\frac{1}{k^p}$ не стремится к нулю, а значит ряд не сходится.\\
Теперь рассмотрим $p > 0$. По доказанному признаку Коши ряд $\sum_{k = 1}^{\infty} \frac{1}{k^p}$ сходится тогда и только тогда, когда сходится ряд 
\[\sum_{k = 1}^{\infty} \frac{2^k}{2^{kp}} = \sum_{k = 1}^{\infty} (2^{1 - p})^k\]
Это геометрическая прогрессия, которая сходится при $2^{1 - p} < 1$, т. е. при $p > 1$.
\end{itemize}

\section{Билет}
\begin{itemize}

\item \textbf{ Подпоследовательность и частичные пределы}
Пусть задана последовательность $\{a_n\}_{n=1}^\infty$ и пусть задана возрастающая последовательность натуральных чисел $n_1 < n_2 < n_3 < \ldots$. Последовательность $b_k = a_{n_k}$ называется подпоследовательностью $\{a_n\}_{n=1}^\infty$.

Число $a \in \mathbb{R}$ называется частичным пределом последовательности $\{a_n\}_{n=1}^\infty$, если выполнено $a = \lim\limits_{k \to \infty}{a_{n_k}}$ для некоторой подпоследовательности $\{a_{n_k}\}_{k=1}^\infty$.

\item \textbf{Верхний и нижний частичные пределы ограниченной последовательности} \\
\textbf{Теорема.} Любая подпоследовательность сходящейся последовательности сходится к пределу этой последовательности. \\
\textbf{Доказательство.}
Рассмотрим последовательность $\{a_n\}_{n=1}^\infty$. Пусть $\lim\limits_{n \to \infty}{a_n} = a$ и пусть $\{a_{n_k}\}_{k=1}^\infty$ - подпоследовательность. По определение предела для каждого $\varepsilon > 0$ найдется номер $N$, для которого $|a_n - a| < \varepsilon$ при $n > N$. Т. к. $1 \leq n$ и $n_{k-1} < n_k$, по индукции получаем, что $k \leq n_k$. Поэтому $|a_{n_k} - a| < \varepsilon$ при $k > N$.

Рассмотрим последовательности $M_n := \sup\limits_{k>n}{a_k}$ и $m_n := \inf\limits_{k>n}{a_k}$. Ясно, что последовательность $M_n$ - не возрастает, а последовательность $m_n$ - не убывает. Поэтому для ограниченной последовательности $\{a_n\}_{n=1}^\infty$ существуют пределы:
\[
    \varlimsup_{n \to \infty}{a_n} := \lim_{n \to \infty}{M_n} 
    ,\ \ \
    \varliminf_{n \to \infty}{a_n} :=  \lim_{n \to \infty}{m_n} 
\]
которые называются соответственно \textbf{верхним} и \textbf{нижним} частичными пределами последовательности $\{a_n\}_{n=1}^\infty$.

\item\textbf{ Теорема о том, что нижний и верхний частичные пределы действительно наименьший и наибольший частичный пределы соответственно}

\textbf{Теорема.} 
Пусть $\{a_n\}_{n=1}^\infty$ - ограниченная последовательность. Тогда $\varlimsup\limits_{n \to \infty}{a_n}$ и $\varliminf\limits_{n \to \infty}{a_n}$  - частичные пределы последовательности $\{a_n\}_{n=1}^\infty$ и любой другой частичный предел принадлежит отрезку $[\varliminf\limits_{n \to \infty}{a_n},\ \varlimsup\limits_{n \to \infty}{a_n}]$.

\textbf{Доказательство.} Покажем, что $M := \varlimsup\limits_{n \to \infty}{a_n}$ - частичный предел. Для этого индуктивно построим подпоследовательность, которая сходится к $\varlimsup\limits_{n \to \infty}{a_n}$. Пусть $n_1 = 1$. Пусть индексы $n_1 < n_2 < \ldots < n_k$ уже построены. Тогда подберем такой номер $n_{k+1} > n_k$, что 
\[
M_{n_k} - \frac{1}{k+1} < 
a_{n_{k+1}} < 
M_{n_k}.
\]
Как подпоследовательность сходящейся последоательности $M_{n_k} \to M$, поэтому по теореме о сходимости зажатой последовательности получаем, что $\lim\limits_{k \to \infty}{a_{n_k}} = M$. Аналогично проверяется, что $\varliminf\limits_{n \to \infty}{a_n}$ - частичный предел.

Пусть теперь $a$ - частичный предел. Это означает, что $a = \lim\limits_{k \to \infty}{a_{n_k}}$ для некоторой подпоследовательности $\{a_{n_k}\}_{k=1}^\infty$. Тогда $m_{n_{k-1}} \leq a_{n_k} \leq M_{n_{k-1}}$. По теореме о переходе к пределу в неравенствах получаем, что $\varliminf\limits_{n \to \infty}{a_n} \leq a \leq \varlimsup\limits_{n \to \infty}{a_n}$.

\item \textbf{Теорема Больцано (следствие из предыдущего пункта)} 

\textbf{Теорема.} 
Во всякой ограниченной последовательности можно найти сходящуюся подпоследовательность.

\item \textbf{Критерий сходимости последовательности в терминах структуры множества частичных пределов.}

\textbf{Теорема.} Ограниченная последовательность сходится тогда и только тогда, когда множество ее частичных пределов состоит из одного элемента.

\textbf{Доказательство.} То, что у сходящейся последовательности есть единственный частичный предел уже проверено ранее.

Предположим, что у ограниченной последовательности $\{a_n\}_{n=1}^\infty$ существует единственный частичный предел. По доказанному, это в частности означает, что 
\[
\varlimsup_{n \to \infty}{a_n} = 
\varliminf_{n \to \infty}{a_n} = 
a
\]
Тогда $m_{n-1} \leq a_n \leq M_{n-1}$ и по теореме о сходимости зажатой последовательности получаем, что $\lim\limits_{n \to \infty}{a_n} = a$.
\end{itemize}
\end{document}