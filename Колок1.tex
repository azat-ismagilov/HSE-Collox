\documentclass[12pt,a4paper]{article}
\usepackage[utf8]{inputenc}
\usepackage[russian]{babel}
\usepackage[OT1]{fontenc}
\usepackage{amsmath}
\usepackage{amsfonts}
\usepackage{amssymb}
\title{Коллоквиум по мат. анализу №1}
\begin{document}
\maketitle
\section{Билет}
\begin{itemize}
\item Рациональные числа - числа вида $\frac{p}{q}$, где $q$ - натуральное число, а $p$ - целое. Считается, что две записи $\frac{p_1}{q_1}$ и  $\frac{p_2}{q_2}$ задают одно и то же рациональное число, если $p_1q_2=p_2q_1$. Обратим внимание на то, что рациональных чисел не достаточно для естественных потребностей математики.

\item Вещественные числа - множество всех бесконечно десятичных дробей вида $\pm a_0a_1a_2...$, где $a_0 \in N \vee {0}, a_j \in {0...9}$ (Записи, в которых с какого-то момента стоят только 9-ки запрещены); \\
Число $\pm 0,000...$ называется нулём и совпадает с числом 0;\\
Нунелевое число:
- положительное, если в его записи стоит знак '+';
- отрицательное, если в его записи стоят знак '-'; \\
В вещественные числа вложены рациональные естественным образом. У вещественных чисел также определены операции сложения и умножения для которых справедливы все их естественные свойства. \\
Отношение порядка у вещественных чисел задано лексикографическим порядком. ($a_0a_1a_2...\leq b_0b_1b_2... \exists k: a_0 = b_0, ... a_{k-1}=b_{k-1}, a_k \leq b_k$), который естественным обращом переносится на отрицательные. \\
Для вещественных чисел определён модуль числа $a$, т.е. такое вещественное число, что $|a| = a$, если $a \geq 0$ и $|a| = -a$, если $a < 0$. Также, для модуля выполняется неравенство треугольника $|a+b| \leq |a| + |b|$. Из неравенства треугольника следует, что $||a|-|b|| \leq |a+b|$. \\
Самое важное свойство - выполняется принцип полноты;

\item Десятичные дроби. Рациональное число может быть представлено в виде конечной или периодической десятичной дроби ($\frac{1}{10} = 0.1; \frac{1}{6} = 0.1(6); \frac{1}{7} = 0.(142857)$. Можно не рассматривать десятичные записи с периодом 9, т.к. $0.(9) = 1$ (Если $0.(9) = x$, то $10x=9+x$ - истина, октуда $x = 1$.

\item Принцип полноты. 
Принцип полноты выполняется, если для произвольных непустых $A$ левее $B$ найдется разделяющий их элемент. \\
Принцип полноты не выполняется для рациоональных чисел. \\
Принцип полноты выполняется на множестве вещественных чисел (теорема).\\
Доказательство: \\
Пусть $A$ и $B$ - непустые множества.  $A$ левее $B$. Если $A$ состоит только из неположительных чисел, а $B$ только из неоотрицательных, то нуль разделяем на $A$ и $B$. Пусть в $A$ имеется положительный элемент, тогда $B$ состоит только из положительных чисел (обратный случай аналогичен). Построим число $c = c_0c_1c_2...$, разделяющее $A$ и $B$. \\
Рассмотрим множество натуральных чисел, с которых начинаются элементы множества $B$. Пусть $b_0$ - наименьшее


\end{itemize}
\end{document}